group-\/project-\/pt2-\/r-\/amin-\/j-\/herup-\/p-\/nipay created by Git\+Hub Classroom \section*{Audio File Management System -\/ Semester Project}

\subsection*{List of Group Members\+:}


\begin{DoxyItemize}
\item Roohan Amin 
\item Pamella Nipay 
\item Josh Herup 
\end{DoxyItemize}

\subsection*{Responsibilities of each group member\+:}


\begin{DoxyItemize}
\item Roohan Amin -\/ User Interaction 
\item Pamella Nipay -\/ File IO 
\item Josh Herup -\/ \hyperlink{classWav}{Wav} Processing 
\end{DoxyItemize}

\subsection*{Description of design\+:}

We coined the term \char`\"{}\+Great Value Audacity\char`\"{} when working on this project, it takes in an 8 bit or 16 bit wav file, and with that wav file has three functionalities. It can edit metadata as well as on to it, manipulate the audio to either echo, noisegate, or normalize it, and it can export existing technical information and metadata to a .csv file. This is all done via a centralized menu in the terminal, where different class objects are called to be implemented into the menu.

\subsection*{Class Diagram}



\subsection*{Challenges while developing}

There was thankfully no trouble in communicating with each other. Our struggles involved extracting and writing the metadata, as well as how to work with the two channels in stereo output. User interaction was of course a breeze, albeit a tedious one. Implementing file\+Write and file\+Read functions proved to be challenging as well. 